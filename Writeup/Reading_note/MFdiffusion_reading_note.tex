\documentclass[10pt,letterpaper]{article}
\usepackage[utf8]{inputenc}
\usepackage{pdflscape}
\usepackage{amsmath}
\usepackage{amsmath}
\usepackage{amsfonts}
\usepackage{fullpage}
\usepackage{amssymb}
\usepackage{threeparttable}
\usepackage[dvipdfmx, hiresbb]{graphicx}
\usepackage[outdir=./]{epstopdf}
\usepackage{natbib}
\usepackage[pdftex]{hyperref}
\hypersetup{colorlinks = true,
                citecolor = {black}}
\usepackage{bbm}
\usepackage{multicol}
\usepackage{longtable}
\usepackage{pgf,pgffor}
\usepackage{tikz}
\setcounter{totalnumber}{8}
\author{Mizuhiro Suzuki}
\title{}
\begin{document}

\maketitle

\tableofcontents

\section{}

\begin{itemize}
  \item Main Matlab file: \texttt{"GMMDiffusion/Main\_models\_1\_3"} for model without $\lambda$ (endorsement effects) and \texttt{"GMMDiffusion/Main\_models\_2\_4"} for model with $\lambda$
\end{itemize}

\section{\texttt{Main\_models\_1\_3.m}}\label{main_models_1_3}

\subsection*{Part 0}
\begin{itemize}
  \item 26: 2 model types: modelType == 1 for $q_N = q_P$ and == 3 for $q_N \ne q_P$
\end{itemize}

\subsection*{Part 2}
\begin{itemize}
  \item 33: Different versions of moments
    (In the paper, they say they use 6 moments, but no version says 6 moments, why?)
\end{itemize}

\subsection*{Part 3}
\begin{itemize}
  \item 49: \texttt{TMonths}: number of months in data of each village
\end{itemize}

\subsection*{Part 4}
\begin{itemize}
  \item 63-69: Parameter grids: although explained in detail in the Supplementary Materials, not shown how they did it
\end{itemize}

\subsection*{Part 5}
\begin{itemize}
  \item 90: Load the network data, which contains adjacency matrices of 43 villages
  \item 95-: For each village,
    \begin{itemize}
      \item 101: Load the leader data: ID and dummy of being a leader or not
      \item 106: Load the take-up data: Only dummy (maybe the same order as the one above?)
      \item 107: \texttt{EmpRate} is the proportion of non-leaders who took up MF
      \item 110: \texttt{inGiant}??? 
        Maybe the biggest cluster in the village?
      \item 113-114: \texttt{d} = degree of network, \texttt{hermits} = households without any connections with other households (?)
      \item 117: Load the covariates of the households (not sure which column corresponds to which variable, maybe in readme file or somewhere?)
      \item 120: Select covariates to use (but in the end they used all the covariates in the data?)
      \item 123-130: Restrict the sample to households belonging to the giant networks?
        \begin{itemize}
          \item 123: Leaders
          \item 124: Take-up
          \item 125: Covariates
          \item 127: Take-up by leaders
          \item 128: Covariates of leaders
          \item 129: Outcome = Take-up by leaders
          \item 130: Covars = Covariates of leaders
            \begin{itemize}
              \item \texttt{Outcome} and \texttt{Covars} will be used to estimate $\beta$
            \end{itemize}
        \end{itemize}
      \item 133-137: Second neighbors (\texttt{sec}) = matrix indicating households whose distance between them is 2.
        Here, $X^2$ means $X * X$.
        (See \href{https://en.wikipedia.org/wiki/Adjacency_matrix#Matrix_powers}{here})
        \begin{itemize}
          \item 134-136: Ruling out households themselves (with two steps, they can always come back to themselves)
          \item 137: Ruling out households with distance 1 (think about a triangle)
        \end{itemize}
    \end{itemize}
\end{itemize}

\subsection*{Part 6}
\begin{itemize}
  \item 142-143: Logistic regression by using leaders to estimate $\beta$
\end{itemize}

\subsection*{Part 7}
\begin{itemize}
  \item 147-160: Calculate the moments in the case where $q_N = q_P$
    \begin{itemize}
      \item 153-160: Calculate moments at each grid of $q_N (= q_P)$
        \begin{itemize}
          \item 156: Call the function \hyperref[divergence_model]{\texttt{divergence\_model}} which calculates moments
        \end{itemize}
    \end{itemize}
  \item 163-178: Calculate moments in the case where $q_N \ne q_P$
    \begin{itemize}
      \item 169-174: Calculate moments at each grid of $(q_N, q_P)$
        \begin{itemize}
          \item 172: Call the function \hyperref[divergence_model]{\texttt{divergence\_model}} which calculates moments
        \end{itemize}
    \end{itemize}
  \item Note: In the calculations here, a matrix $D$ is obtained.
    This is a length($q_N$ grids)-by-length($q_P$ grids) matrix, and each element is a (\# villages)-by-(\# moments) matrix.
\end{itemize}

\subsection*{Part 8}
\begin{itemize}
  \item 188-194: Select if just obtaining point estimates or standard errors through bootstrap
    \begin{itemize}
      \item bootstrap == 0: point estimates
      \item bootstrap == 1: standard errors through bootstrap (1000 times)
    \end{itemize}
  \item 197-207: Select if two step optimal weights are used or not
    \begin{itemize}
      \item twoStepOptimal == 1: 
        \begin{itemize}
          \item 202: With some guess of $q_N (= 0.09)$ and $q_P (= 0.45)$, obtain moments by the function \texttt{divergence\_model}
          \item 203: Calculate the outer product of the moments, divided by the number of villages
          \item 204: Take the inverse of the above expression, which will be the weight in the second step
          \item \textbf{Note}: I guess usually people 
            (i) solve the GMM problem with an identity matrix as a weight to get the consistent estimate of parameters, and 
            (ii) get the weight in the way described above.
            Here, somehow authors set $q_N$ and $q_P$ and calculate the weight based on these parameters.
            If these are not derived by consistent estimates, then the weight is inconsistent and thus the estimates in the second step will be biased as well.
        \end{itemize}
      \item twoStepOptimal == 0: use an identity matrix as the weight in the ``second'' step
    \end{itemize}
  \item 211-216: Obtain a new 4-dimension (length($q_N$ grids)-by-length($q_P$ grids)-by-(\# villages)-by-(\# moments))   $Dnew$ from a nested matrix $D$
  \item 223-259: Bootstrap (note: here in the bootstrap, randomly generated village weights are used, not that villages are randomly chosen with replacement)
    \begin{itemize}
      \item 226-231: Generate bootstrap weights 
        \begin{itemize}
          \item bootstrap == 0 (ie. for point estimate): same weights for all villages
          \item bootstrap == 1 (ie. for bootstrap standard errors): randomly generated weights follow exponential distribution (normalized so that the sum of weights is $1$)
        \end{itemize}
      \item 236-244: Calculate the criterion function for each ($q_N, q_P$) grid
        \begin{itemize}
          \item 240: Calculate weighted moments 
          \item 242: Calculate the criterion function
        \end{itemize}
      \item 246-254: Derive the parameter values that minimize the objective function and store the results
    \end{itemize}
\end{itemize}


\section{\texttt{divergence\_model.m}}\label{divergence_model}

``This computes the deviation of the empirical moments from the simulated ones''

\subsection*{Arguments}
\begin{itemize}
  \item X: adjacency matrix
  \item Z: covariate
  \item Betas: estimates of $\beta$'s 
  \item leaders: vector indicating leaders
  \item TakeUp: vector indicating who took up microfinance
  \item Sec: matrix indicating second neighbors (= households with distance two)
  \item theta: parameter values ($q_N$, $q_P$)
  \item m: number of moments
  \item S: number of simulations
  \item T: number of months
  \item EmpRate: proportion of non-leaders who took up MF
  \item version: specification of which moments to use
\end{itemize}

\subsection*{Main part}
\begin{itemize}
  \item 23-39: Calculate empirical and simulated moments
    \begin{itemize}
      \item 25: Calculate empirical moments, using a function \hyperref[moments]{\texttt{moments}}
      \item 28-32: Calculate simulated moments
        \begin{itemize}
          \item 30: Simulate the take-up of microfinance, using a function \hyperref[diffusion_model]{\texttt{diffusion\_model}}
          \item 31: Calculate simulated moments based on the simulated take-ups, using a function \hyperref[moments]{\texttt{moments}}
        \end{itemize}
      \item 36: Take the different of empirical and simulated moments
    \end{itemize}
\end{itemize}

\section{\texttt{moments.m}}\label{moments}

\subsection*{Arguments}
\begin{itemize}
  \item X: adjacency matrix
  \item leaders: vector indicating leaders
  \item infected: vector indicating who took up microfinance
  \item Sec: matrix indicating second neighbors (= households with distance two)
  \item j: village index
  \item version: specification of which moments to be used
\end{itemize}

\subsection*{Main part}
\begin{itemize}
  \item 3: Declare a variable ``netstats'' which contains information of networks in each village (for the function \texttt{persistent}, see \href{https://www.mathworks.com/help/matlab/ref/persistent.html}{here})
  \item 7-49: If a variable ``netstats'' is already defined, then almost skip the parts, but if not yet, then calculate each statistics of the networks (7-43)
    \begin{itemize}
      \item 8: Use a function \hyperref[breadthdistRAL]{\texttt{breadthdistRAL}} to calculate ??? (R: ???, D: ???)
      \item 
    \end{itemize}
  \item 52-: Moments under different versions
    \begin{itemize}
      \item 53-88: Case1 (moments (2)-(6) in the paper):
        \begin{itemize}
          \item 54-62: ``Fraction of nodes that have no taking neighbors but are takers themselves'' (moment (2) in the paper)
            \begin{itemize}
              \item 56: Number of infected neighbors: \\
                \texttt{ones(N,1)*infected'}: a matrix with indicators of infected households in each row \\
                $\Rightarrow$ \texttt{(ones(N,1)*infected').*X}: a matrix with indicators of infected neighbors for each household \\
                $\Rightarrow$ \texttt{sum((ones(N,1)*infected').*X, 2)}: number of infected neighbors for each household
              \item 58-59: If there is any household who are linked with other households (netstats(j).degree $>$ 0) but has no infected neighbors (infected Neighbors == 0), 
                calculate the fraction $\frac{\text{\# HH who are linked with other households, don't have any infected neighbors, but are infected themselves}}{\text{\# HH who are linked with other households and don't have any infected neighbors}}$
              \item 60-61: If there is no such household in the village, then just let the moment be $0$ (since the denominator of the above fraction will be $0$ in this case)
            \end{itemize}
          \item 64-69: ``Fraction of individuals that are infected in the neighborhood of infected leaders stats(1) = 0'' (moment (3) in the paper)
            \begin{itemize}
              \item If \texttt{sum(netstats(j).neighborOfInfected) $>$ 0} (???), calculate the fraction (???)
              \item Otherwise, just let the moment be $0$ 
            \end{itemize}
          \item 71-76: ``Fraction of individuals that are infected in the neighborhood of non-infected leaders)'' (moment (4) in the paper)
            \begin{itemize}
              \item If \texttt{sum(netstats(j).neighborOfNonInfected) $>$ 0} (???), calculate the fraction (???)
              \item Otherwise, just let the moment be $0$ 
            \end{itemize}
          \item 78-82: ``Covariance of individuals taking with share of neighbors taking'' (moment (5) in the paper)
            \begin{itemize}
              \item 79: NonHermits: Indicator of non-isolated households 
              \item 80: ShareofTakingNeighbors: For each non-isolated household, \\
                $\frac{\text{\# infected neighbors}}{\text{\# neighbors}}$
              \item 81: NonHermitTakers: Indicator of non-isolated households who took up
              \item 82: $\text{moment} = \frac{\sum_{i: \text{non-isolated}} (\text{Take-up})_i \times (\text{Share of taking neighbors})_i}{\text{\# Non-isolated households}}$
            \end{itemize}
          \item 82-88: ``Covariance of individuals taking with share of second neighbors taking'' (moment (6) in the paper)
            \begin{itemize}
              \item 86: infectedSecond: number of infected second neighbors
              \item 87: ShareofSecond: Share of infected second neighbors for non-isolated households, \\
                $\frac{\text{\# infected second neighbors}}{\text{\# first neighbors}}$ (** Why is the denominator about first neighbors? **)
              \item 88: $\text{moment} = \frac{\sum_{i: \text{non-isolated}} (\text{Take-up})_i \times (\text{Share of taking second neighbors})_i}{\text{\# Non-isolated households}}$
            \end{itemize}
        \end{itemize}
      \item 91-112: Case2 (moments (2), (5), and (6) in the paper):
        \begin{itemize}
          \item 92-100: ``Fraction of nodes that have no taking neighbors but are takers themselves'' (moment (2) in the paper)
          \item 102-106: ``Covariance of individuals taking with share of neighbors taking'' (moment (5) in the paper)
          \item 110-112: ``Covariance of individuals taking with share of second neighbors taking'' (moment (6) in the paper)
        \end{itemize}
      \item 115-141: Case3 (moments (2), (5), and (6) in the paper, same as case 2, but purged of leader injection points):
        \begin{itemize}
          \item 117: a variable that denotes whether a node is a leader
        \end{itemize}
      \item 143-169: Case4 (moments (2), (5), and (6) in the paper, same as case 2, but purged of all leader points):
        \begin{itemize}
          \item 117: a variable that denotes whether a node is a leader
          \item 166-167: Moment (6), but second neighbors who are leaders are not taken into account (see line 167)
        \end{itemize}
    \end{itemize}
\end{itemize}

\section{\texttt{breadthdistRAL.m}}\label{breadthdistRAL}
TBD...

\section{\texttt{diffusion\_model.m}}\label{diffusion_model}

\subsection*{Arguments}
\begin{itemize}
  \item parms: parameters ($q_N$, $q_P$)
  \item Z: covariate
  \item Betas: estimates of $\beta$'s 
  \item X: adjacency matrix
  \item leaders: vector indicating leaders
  \item j: village index
  \item T: number of months
  \item EmpRate: proportion of non-leaders who took up MF
\end{itemize}

\subsection*{Main part}

\begin{itemize}
  \item 7-11: Prepare arrays to be updated in the simulation below
    \begin{itemize}
      \item 7: \texttt{infected}: people who are already infected and newly infected (initial value = false for everyone)
      \item 8: \texttt{infectedbefore}: people who are already infected (initial value = false for everyone)
      \item 9: \texttt{contagiousbefore}: people who are already informed (initial value = false for everyone)
      \item 10: \texttt{contagious}: people who are newly informed and already informed (?) (initial value = true only for leaders)
      \item 11: \texttt{dynamicInfection}: vector that tracks the infection rate for the number of periods it takes place
    \end{itemize}
  \item 16-39: Loop to simulate the diffusion for $T$ periods
    \begin{itemize}
      \item 20-27: Step 1: Take-up decision based on newly informed
        \begin{itemize}
          \item 21: Probability of take-up (calculated for everyone, both non-infected and already-infected)
          \item 22: Updated infected households: \\
            The condition \texttt{($\sim$contagiousbefore \& contagious \& x(:,t) $<$ LOGITprob)} is satisfied if:
            \begin{itemize}
              \item not contagious before, and 
              \item newly informed, and
              \item the random number generated is smaller than the LOGITprob calculated in line 21.
            \end{itemize}
          \item 25: Update \texttt{infectedbefore}
          \item 26: Update \texttt{contagiousbefore}
          \item 27: \texttt{C}: number of informed households
        \end{itemize}
      \item 29-31: Step 2: Information flows
        \begin{itemize}
          \item 30: \texttt{transmitPROB}: Vector of probability of transmission (individual specific, depending on informed and on take-up)
            \begin{itemize}
              \item If informed \& take-up, probability = $q_P$
              \item If informed \& not take-up, probability = $q_N$
            \end{itemize}
          \item 31: \texttt{contagionlikelihood}: The probability to information flowing from informed households to neighbors (no matter whether those neighbors are already informed or not) \\
            eg) Suppose that
            $X = \begin{pmatrix}
              0 & 1 & 1 & 1 \\
              1 & 0 & 0 & 1 \\
              1 & 0 & 0 & 0 \\
              1 & 1 & 0 & 0
            \end{pmatrix}$,
            $\text{transmitPROB} = \begin{pmatrix} q_P \\ 0 \\ 0 \\ q_N \end{pmatrix}$,
            and $\text{contagious} = \begin{pmatrix} 1 \\ 0 \\ 0 \\ 1 \end{pmatrix}$.
            Then, 

            \begin{align*}
              &\quad \text{\texttt{X(contagious,:).*(transmitPROB(contagious)*ones(1,N))}} \\
              &= 
            \begin{pmatrix}
              0 & 1 & 1 & 1 \\
              1 & 1 & 0 & 0
            \end{pmatrix} .* 
            \begin{pmatrix}
              q_P & q_P & q_P & q_P \\
              q_N & q_N & q_N & q_N
            \end{pmatrix} \\
              &= \begin{pmatrix}
              0 & q_P & q_P & q_P \\
              q_N & q_N & 0 & 0
            \end{pmatrix}
            \end{align*}        
        \end{itemize}
      \item 33-36: Step 3: Simulate newly informed households
        \begin{itemize}
          \item 34: Update \texttt{contagious}:
            \begin{itemize}
              \item \texttt{(contagionlikelihood $>$ rand(C,N))}: simulation of whether households are informed from already informed neighbors
              \item \texttt{((contagionlikelihood $>$ rand(C,N))'*ones(C,1) $>$ 0)}: informed at least from one neighbor
            \end{itemize}
          \item 36: Update \texttt{dynamicInfection} to keep track of the fraction of take-up households
        \end{itemize}
    \end{itemize}
\end{itemize}




\clearpage
%\bibliographystyle{apalike}
%\bibliography{audit}

\end{document}

